\documentclass[a4paper, 12pt]{article}

\usepackage[T2A]{fontenc}
\usepackage[utf8]{inputenc}
\usepackage[english, russian]{babel}
\usepackage{url}
\usepackage{booktabs}
\usepackage{nicefrac}
\usepackage{microtype}
\usepackage{lipsum}
\usepackage{graphicx}
\usepackage{subfig}
\usepackage[square,sort,comma,numbers]{natbib}
\usepackage{doi}
\usepackage{multicol}
\usepackage{multirow}
\usepackage{tabularx}
\usepackage{tabulary}
\usepackage{pifont}
\usepackage{colortbl}

\usepackage{tikz}
\usetikzlibrary{matrix}
% Algorithms
\usepackage{algorithm}
\usepackage{algpseudocode}
\makeatletter
\renewcommand{\ALG@name}{Алгоритм}
\makeatother
%% Шрифты
\usepackage{euscript} % Шрифт Евклид
\usepackage{mathrsfs} % Красивый матшрифт
% \usepackage{extsizes}
\usepackage{makecell} % diaghead in a table
\usepackage{amsmath,amsfonts,amssymb,amsthm,mathtools,dsfont}
\usepackage{icomma}

\usepackage[center]{titlesec}
\usepackage{amsthm}
\usepackage{amssymb}
\usepackage{amsmath}
\usepackage{color}
\usepackage{wrapfig}
\usepackage{caption}
% \usepackage{subcaption}
\usepackage{indentfirst}
\usepackage[utf8]{inputenc} % allow utf-8 input
\usepackage[T1]{fontenc}    % use 8-bit T1 fonts
\usepackage{hyperref}       % hyperlinks
\usepackage{url}            % simple URL typesetting
\usepackage{booktabs}       % professional-quality tables
\usepackage{amsfonts}       % blackboard math symbols
\usepackage{nicefrac}       % compact symbols for 1/2, etc.
\usepackage{microtype}      % microtypography
\usepackage{xcolor}         % colors
\usepackage{stackrel}
\usetikzlibrary{matrix,arrows.meta,quotes,positioning}
\usetikzlibrary{calc}

\usepackage{hyperref}
\hypersetup{
	unicode=true,
	pdftitle={Generative causal inference approach to BCI analysis},
	pdfsubject={q-bio.NC, q-bio.QM},
	pdfauthor={Vladimirov Eduard Anatolevich},
	pdfkeywords={First keyword, Second keyword, More},
	colorlinks=true,
	linkcolor=black,        % внутренние ссылки
	citecolor=green,        % на библиографию
	filecolor=magenta,      % на файлы
	urlcolor=blue           % на URL
}

\usepackage{enumitem} % Для модификаций перечневых окружений
\usepackage{etoolbox}
\expandafter\patchcmd\csname\string\algorithmic\endcsname{\itemsep\z@}{\itemsep=1.5mm}{}{}

\usepackage{geometry}
\geometry{left=3cm}
\geometry{right=1.5cm}
\geometry{top=2.0cm}
\geometry{bottom=2.0cm}
\setlength\parindent{1.25cm}    % Устанавливает длину красной строки 15pt
\linespread{1.3}             % Коэффициент межстрочного интервала

\usetikzlibrary{arrows.meta}
\usetikzlibrary{quotes}

\def\ltrn{\mathcal{L}_{\mathrm{train}}}
\def\lval{\mathcal{L}_{\mathrm{val}}}

\newcommand\argmin{\mathop{\arg\min}}
\newcommand{\hchi}{\hat{\boldsymbol{\chi}}}
\newcommand{\hphi}{\hat{\boldsymbol{\varphi}}}
\newcommand{\bchi}{\boldsymbol{\chi}}
\newcommand{\A}{\mathcal{A}}
\newcommand{\B}{\mathcal{B}}
\newcommand{\x}{\mathbf{x}}
\newcommand{\hx}{\hat{x}}
\newcommand{\hy}{\hat{y}}
\newcommand{\M}{\mathcal{M}}
\newcommand{\N}{\mathcal{N}}
\newcommand{\R}{\mathbb{R}}
\newcommand{\p}{p(\cdot)}
\newcommand{\q}{q(\cdot)}
\newcommand{\uu}{\mathbf{u}}
\newcommand{\vv}{\mathbf{v}}
\newcommand{\bz}{\mathbf{z}}
\newcommand{\bx}{\mathbf{x}}
\newcommand{\by}{\mathbf{y}}
\newcommand{\bv}{\mathbf{v}}
\newcommand{\bw}{\mathbf{w}}
\newcommand{\ba}{\mathbf{a}}
\newcommand{\bb}{\mathbf{b}}
\newcommand{\bff}{\mathbf{f}}
\newcommand{\bh}{\mathbf{h}}
\newcommand{\bl}{\mathbf{l}}
\newcommand{\bp}{\mathbf{p}}
\newcommand{\bq}{\mathbf{q}}
\newcommand{\bs}{\mathbf{s}}
\newcommand{\bt}{\mathbf{t}}
\newcommand{\bu}{\mathbf{u}}
\newcommand{\bT}{\mathbf{T}}
\newcommand{\bX}{\mathbf{X}}
\newcommand{\bZ}{\mathbf{Z}}
\newcommand{\bS}{\mathbf{S}}
\newcommand{\bI}{\mathbf{I}}
\newcommand{\bH}{\mathbf{H}}
\newcommand{\bW}{\mathbf{W}}
\newcommand{\bY}{\mathbf{Y}}
\newcommand{\bU}{\mathbf{U}}
\newcommand{\bQ}{\mathbf{Q}}
\newcommand{\bP}{\mathbf{P}}
\newcommand{\bA}{\mathbf{A}}
\newcommand{\bB}{\mathbf{B}}
\newcommand{\bC}{\mathbf{C}}
\newcommand{\bE}{\mathbf{E}}
\newcommand{\bF}{\mathbf{F}}
\newcommand{\bK}{\mathbf{K}}
\newcommand{\bV}{\mathbf{V}}
\newcommand{\bsigma}{\boldsymbol{\sigma}}
\newcommand{\bSigma}{\boldsymbol{\Sigma}}
\newcommand{\bomega}{\boldsymbol{\omega}}
\newcommand{\btheta}{\boldsymbol{\theta}}
\newcommand{\bgamma}{\boldsymbol{\gamma}}
\newcommand{\bdelta}{\boldsymbol{\delta}}
\newcommand{\bPsi}{\boldsymbol{\Psi}}
\newcommand{\bpsi}{\boldsymbol{\psi}}
\newcommand{\bxi}{\boldsymbol{\xi}}
\newcommand{\bmu}{\boldsymbol{\mu}}
\newcommand{\bzeta}{\boldsymbol{\zeta}}
\newcommand{\blambda}{\boldsymbol{\lambda}}
\newcommand{\beps}{\boldsymbol{\varepsilon}}
\newcommand{\bZeta}{\boldsymbol{Z}}
% mathcal
\newcommand{\cX}{\mathcal{X}}
\newcommand{\cY}{\mathcal{Y}}
\newcommand{\cW}{\mathcal{W}}
\newcommand{\dH}{\mathds{H}}
\newcommand{\dR}{\mathds{R}}
% transpose
\newcommand{\T}{^{\mathsf{T}}}
% command to strike out text
\newcommand{\stkout}[1]{\ifmmode\text{\sout{\ensuremath{#1}}}\else\sout{#1}\fi}
\renewcommand{\epsilon}{\ensuremath{\varepsilon}}
\renewcommand{\phi}{\ensuremath{\varphi}}
\renewcommand{\kappa}{\ensuremath{\varkappa}}
\renewcommand{\le}{\ensuremath{\leqslant}}
\renewcommand{\leq}{\ensuremath{\leqslant}}
\renewcommand{\ge}{\ensuremath{\geqslant}}
\renewcommand{\geq}{\ensuremath{\geqslant}}
\renewcommand{\emptyset}{\varnothing}
\renewcommand{\baselinestretch}{1}

\newcommand{\vect}[1]{\boldsymbol{\mathbf{#1}}}
\newtheorem{theorem}{Theorem}[section]
\newtheorem{corollary}{Corollary}[theorem]
\newtheorem{lemma}[theorem]{Lemma}
\newtheorem{proposition}[theorem]{Proposition}
\newtheorem{assumption}[theorem]{Assumption}

\newtheorem{Th}{Теорема}
\newtheorem{Def}{Определение}
\newenvironment{Proof} % имя окружения
{\par\noindent{\bf Proof:}} % команды для \begin
{\hfill$\scriptstyle\blacksquare$} % команды для \end
\newtheorem{Assumption}{Предположение}
\newtheorem{Corollary}{Следствие}
\newtheorem{problem}{Problem}

\graphicspath{{../figures/}}

\renewcommand{\abstractname}{Аннотация}
\renewcommand{\abstractname}{Аннотация}

\title{Генеративный причинно-следственный подход к анализу данных ИМК}

\author{Владимиров Эдуард \\
	\texttt{vladimirov.ea@phystech.edu} \\
	\And
	Стрижов Вадим \\
	\texttt{strijov@phystech.edu}
}
\date{\today}

\begin{document}
	\maketitle
	
	\begin{abstract}
		В данной работе предлагается генеративный причинно-следственный подход к анализу взаимодействия биосигналов, в частности электроэнцефалограммы (ЭЭГ) и инерциальных данных. Главная цель --- определить и количественно оценить причинно-следственные связи между мозговой активностью и движениями, опираясь на три взаимодополняющие методологии: геометрическое представление сигналов на римановых многообразиях, физически-информированные нейронные сети (PINN) для реалистичного моделирования кинематики, и меры информационной теории и алгоритмы причинного анализа. Объединение этих подходов даёт возможность не только выявлять временные и нелинейные зависимости в системе <<мозг--тело>>, но и строить синтетические выборки для валидации гипотез. Предложенный фреймворк повышает точность и интерпретируемость результатов. Его работа проиллюстрирована на датасете игры с теннисом.

	\end{abstract}
	
	
	\keywords{ЭЭГ \and ИИМ \and сходящееся перекрёстное отображение \and взаимная информация \and причинно-следственный вывод} 
	
	\section{Введение}
	
	В последние годы наблюдается стремительный рост интереса к исследованию взаимодействия между мозговой активностью и двигательными действиями человека. С этой целью широко применяются методы анализа электроэнцефалографических (ЭЭГ) сигналов и инерциальных данных (ИИМ, инерциальный измерительный модуль), поскольку они позволяют регистрировать, соответственно, нейронную активность и механику движений в реальном времени. Выявление причинно-следственных связей между ЭЭГ и ИИМ играет важнейшую роль в области нейробиологии, медицины, спортивной науки и технологий Brain-Computer Interface (BCI). Традиционные подходы к анализу таких данных — например, методы корреляции или линейной регрессии — зачастую оказываются недостаточными для точного описания сложных нелинейных взаимодействий в системе <<мозг--тело>>.
	
	Чтобы преодолеть указанные ограничения, в настоящей работе рассматривается \textbf{генеративный причинно-следственный подход} (Generative Causal Inference), сочетающий несколько концептуальных направлений:
	
	\begin{itemize}
		\item \textbf{Моделирование сигналов на многообразиях.}
		ЭЭГ-данные представляют собой высокоразмерный временной ряд, источник которого ~--- множество независимых (или частично независимых) нейронных процессов. Движения, в свою очередь, могут быть представлены на пространстве кинематических параметров, где измерения акселерометра и гироскопа описывают динамику конечностей. Использование геометрических методов и концепции римановой геометрии (Riemannian geometry) позволяет корректно учитывать внутреннюю структуру и ковариации в данных, избегая упрощающих линейных допущений.
		
		\item \textbf{Физически-информированные нейронные сети (PINN).}
		Для описания динамики движений и сигналов ИИМ всё чаще применяются Physics-Informed Neural Networks. В отличие от стандартного глубокого обучения, PINN встраивает уравнения классической механики (или другой физической модели) прямо в функцию потерь. Это даёт возможность строить более реалистичные (и интерпретируемые) модели движения, которые затем могут использоваться для проверки гипотез о причинно-следственных связях с мозговой активностью.
		
		\item \textbf{Причинно-следственный анализ на базе информационных мер.}
		Использование мер информационной теории, таких как взаимная информация (Mutual Information, MI) и дивергенция Кульбака--Лейблера (Kullback--Leibler divergence, KL), даёт статистически обоснованные способы выявлять и количественно оценивать степень зависимости между сигнальными распределениями. В частности, инструменты вроде Convergent Cross Mapping (CCM) или Probabilistic CCM позволяют проверить, какие каналы ЭЭГ <<вызывают>> изменения ИИМ, или что связь обусловлена иными факторами.
	\end{itemize}
	
	Объединение подходов моделирования с методами причинно-следственного анализа  лежит в основе предлагаемого здесь \textbf{генеративного причинно-следственного вывода}. Это позволяет не только установить факт корреляции или временной зависимости между сигналами, но и даёт возможность:
	
	\begin{itemize}
		\item оценивать устойчивость данных выводов при варьировании параметров модели;
		\item генерировать синтетические выборки, отражающие ключевые особенности реальных записей ЭЭГ и ИИМ;
		\item учитывать сложность биологических систем, проявляющуюся в нелинейных и многокомпонентных связях, которые не могут быть адекватно описаны классическими линейными моделями.
	\end{itemize}
	
	Таким образом, целью данной работы является разработка единого методологического фреймворка, который объединяет:
	\begin{enumerate}
		\item геометрические и физические представления о природе исследуемых сигналов,
		\item продвинутые методы генерации и анализа данных,
		\item алгоритмы выявления причинно-следственных связей.
	\end{enumerate}
	
	В дальнейшем мы продемонстрируем, как такой подход может применяться на практике при анализе когнитивных и моторных данных, а также обсудим преимущества и текущие ограничения.
	
	\section{Теоретическая часть}
	
	\subsection{Общие обозначения}
	
	Рассмотрим два временных ряда: 
	\[
	\mathbf{X}(t) \in \mathbb{R}^{N \times d_X}, 
	\quad
	\mathbf{Y}(t) \in \mathbb{R}^{N \times d_Y},
	\quad
	t = 1,2,\dots,T.
	\]
	
	Пусть $p(\mathbf{X}, \mathbf{Y})$ обозначает совместное распределение этих данных.
	
	Наша цель --- проанализировать причинно-следственные связи между $\mathbf{X}$ и $\mathbf{Y}$, то есть понять, существует ли (и насколько выражена) направленная зависимость:
	\[
	\mathbf{X} \;\longrightarrow\; \mathbf{Y},
	\quad
	\text{или}
	\quad
	\mathbf{Y} \;\longrightarrow\; \mathbf{X}.
	\]
	
	\subsection{Независимый анализ компонент (ICA)}
	
	При наличии высокоразмерных сигналов $\mathbf{X}(t)$, содержащих множество смешанных источников, часто вводится модель:
	\[
	\mathbf{X}(t) = A\,\mathbf{S}(t),
	\]
	где $\mathbf{S}(t)\in\mathbb{R}^{d_S}$ --- вектор независимых компонент, а $A$ --- постоянная матрица смешения. Предполагается, что координаты $\mathbf{S}(t)$ статистически независимы:
	\[
	p(\mathbf{S}) \;=\; \prod_{k=1}^{d_S} p\bigl(S_k\bigr).
	\]
	Задача независимого анализа компонент сводится к оценке $\widehat{A}^{-1}$ (или непосредственно $A$) так, чтобы 
	\[
	\widehat{\mathbf{S}}(t) \;=\; \widehat{A}^{-1}\,\mathbf{X}(t)
	\]
	максимизировало независимость компонент. Для измерения независимости используют различные функции (энтропию, взаимную информацию и др.). Итоговые независимые компоненты $\widehat{\mathbf{S}}(t)$ далее могут анализироваться вместо $\mathbf{X}(t)$, что часто упрощает выявление причинно-следственных связей, особенно в ЭЭГ-исследованиях.
	
	\subsection{Физически-информированные нейронные сети (PINN)}
	
	Сигналы $\mathbf{Y}(t)$ могут удовлетворять законам механики, описываемым уравнениями:
	\[
	\frac{d^2 \mathbf{r}(t)}{dt^2} \;=\; \mathbf{f}\!\bigl(\mathbf{r}(t), \dot{\mathbf{r}}(t), \theta\bigr),
	\]
	где $\mathbf{r}(t) \in \mathbb{R}^m$ --- координаты (или углы) исследуемого объекта во времени, а $\theta$ --- набор физических параметров (масса, длины звеньев и т.\,п.). При использовании модели PINN мы строим приближающую функцию
	\[
	\mathbf{u}(t,\boldsymbol{\Theta}) \;\approx\; \mathbf{r}(t),
	\]
	где $\boldsymbol{\Theta}$ --- обучаемые веса нейронной сети. В функцию потерь $\mathcal{L}(\boldsymbol{\Theta})$ помимо ошибки на измеренных данных $\mathbf{Y}(t)$ (которая соответствует ускорению или угловой скорости) добавляют слагаемое, обеспечивающее физическую согласованность:
	\[
	\mathcal{L}(\boldsymbol{\Theta})
	\;=\;
	\sum_{t}\Bigl\|\ddot{\mathbf{u}}(t,\boldsymbol{\Theta})-\mathbf{f}\!\bigl(\mathbf{u}(t,\boldsymbol{\Theta}),\dot{\mathbf{u}}(t,\boldsymbol{\Theta}),\theta\bigr)\Bigr\|^2
	\;+\;
	\sum_{t}\Bigl\|\ddot{\mathbf{u}}(t,\boldsymbol{\Theta})-\mathbf{Y}(t)\Bigr\|^2,
	\]
	где первая сумма отвечает за удовлетворение уравнениям движения, а вторая --- за согласие с наблюдениями из IMU.
	
	При введении стохастических факторов (например, случайных начальных условий) получается вероятностная модель
	\[
	p_{\boldsymbol{\Theta}}(\mathbf{Y}) 
	\;\approx\;
	\prod_{t} p_{\boldsymbol{\Theta}}(\mathbf{Y}(t) \,\vert\, \mathbf{u}(t,\boldsymbol{\Theta})),
	\]
	позволяющая учесть изменчивость движений при неизменной задаче.
	
	\subsection{Метод сходящихся перекрёстных отображений (CCM)}
	
	Для проверки гипотезы о том, что $\mathbf{X}(t)$ причинно влияет на $\mathbf{Y}(t)$, используют процедуру Convergent Cross Mapping. Вводится теневое вложение:
	\[
	M_{X,t} 
	\;=\;
	\bigl(X_t,\;X_{t-\tau},\;\dots,\;X_{t-(E-1)\tau}\bigr),
	\]
	где $E$ --- размерность вложения, а $\tau$ --- временной лаг. Аналогично задаётся $M_{Y,t}$. Если вектор $M_{X,t}$ хорошо \emph{реконструирует} $\mathbf{Y}_t$, то есть
	\[
	\widehat{\mathbf{Y}}_t 
	\;=\; 
	\sum_{i=1}^{k} w_i \,\mathbf{Y}_{n_i},
	\]
	где $n_i$ --- индексы ближайших соседей $M_{X,t}$ в пространстве $M_X$, то считается, что есть динамическая причинная связь $\mathbf{X}\to \mathbf{Y}$. Значение 
	\[
	\rho_{X\to Y}
	\;=\;
	\mathrm{corr}\!\Bigl(\{\widehat{\mathbf{Y}}_t\},\{\mathbf{Y}_t\}\Bigr)
	\]
	растёт при увеличении размера библиотеки, если $\mathbf{X}\to \mathbf{Y}$ действительно имеет место.
	
	\subsection{Вероятностный метод сходящихся перекрёстных отображений}
	
	Вместо единственного предсказания $\widehat{\mathbf{Y}}_t$ можно рассматривать \emph{условное распределение} 
	\[
	p_L\bigl(\mathbf{Y}_t \,\big\vert\, M_{X,t}\bigr),
	\]
	оценённое на выборке размера $L$. По аналогии с CCM, ближайшие соседи используются для построения вероятностной аппроксимации:
	\[
	p_L\bigl(\mathbf{Y}_t \,\big\vert\, M_{X,t}\bigr)
	\;=\;
	\sum_{i\in N_L(t)} \alpha_i \,\delta\bigl(\mathbf{Y}-\mathbf{Y}_{n_i}\bigr),
	\]
	где $N_L(t)$ --- множество индексов соседей, а $\alpha_i$ --- веса, зависящие от расстояния в $M_X$. Тогда для оценки причинности рассматривают, например, величину
	\[
	I_L\bigl(\mathbf{X}\!\to\!\mathbf{Y}\bigr)
	\;=\;
	\sum_{t}
	D_{\mathrm{KL}}\Bigl(
	p_L\bigl(\mathbf{Y}_t \,\big\vert\, M_{X,t}\bigr) 
	\;\big\|\;
	p(\mathbf{Y}_t)
	\Bigr),
	\]
	или аналогичную меру, связанную с \(\mathrm{MI}\) (взаимной информацией). При $L\to\infty$ распределение $p_L(\mathbf{Y}_t|M_{X,t})$ должно сходиться к истинному условному $p(\mathbf{Y}_t|M_{X,t})$, давая более строгую вероятностную трактовку динамических причинных связей.
	
	\subsection{Ядерное сглаживание}
	
	Для набора $\{M_{X,t}\}_{t=1}^T$ можно ввести эмпирическую плотность с помощью ядерного метода. Пусть $K(\cdot)$ --- некая ядровая функция (напр.\,гауссова):
	\[
	\widehat{p}_X(u)
	\;=\;
	\frac{1}{T h_X^{m_X}}
	\sum_{t=1}^T
	K\!\Bigl(\frac{\|u - M_{X,t}\|}{h_X}\Bigr),
	\]
	где $h_X>0$ --- параметр сглаживания, а $m_X=\dim(\mathcal{M}_X)$ --- формальная размерность пространства вложения. Аналогично для $M_{Y,t}$ определяем
	\[
	\widehat{p}_Y(v)
	\;=\;
	\frac{1}{T h_Y^{m_Y}}
	\sum_{t=1}^T
	K\!\Bigl(\frac{\|v - M_{Y,t}\|}{h_Y}\Bigr).
	\]
	Таким образом, каждому многообразию $\mathcal{M}_X$ и $\mathcal{M}_Y$ соответствует своя оценка плотности, отражающая эмпирическое распределение вложенных точек.
	
	\section{Постановка задачи}
	
	Пусть \( \mathbf{X}(t) = \{X_1(t), X_2(t), \ldots, X_{n_x}(t)\} \) и \( \mathbf{Y}(t) = \{Y_1(t), Y_2(t), \ldots, Y_{n_y}(t)\} \) — два набора многомерных временных рядов, наблюдаемых в моменты времени \( t = 1, \ldots, T \). Нас интересует выявление причинных связей между \( \mathbf{X}(t) \) и \( \mathbf{Y}(t) \), т.е. такие пары связей, где переменные из \( \mathbf{X}(t) \) причинно влияют на переменные из \( \mathbf{Y}(t) \), или наоборот.
	
	Необходимо определить направленные причинные связи:
	1. \( X_i(t-\tau) \to Y_j(t) \) для \( i = 1, \ldots, n_x \), \( j = 1, \ldots, n_y \), и лагов \( \tau \geq 0 \),
	2. \( Y_j(t-\tau) \to X_i(t) \) для \( i = 1, \ldots, n_x \), \( j = 1, \ldots, n_y \), и лагов \( \tau \geq 0 \).
	
	Предполагаем, что многомерные временные ряды \( \mathbf{X}(t) \) и \( \mathbf{Y}(t) \) генерируются следующим образом:
	\[
	X_i(t) = f_i(\text{Pa}_{X_i}(t), \varepsilon_{X_i}(t)),
	\]
	\[
	Y_j(t) = g_j(\text{Pa}_{Y_j}(t), \varepsilon_{Y_j}(t)),
	\]
	где:
	- \( \text{Pa}_{X_i}(t) \subseteq \{Y_1(t-\tau), \ldots, Y_{n_y}(t-\tau)\} \) — множество родителей переменной \( X_i(t) \) из \( \mathbf{Y} \),
	- \( \text{Pa}_{Y_j}(t) \subseteq \{X_1(t-\tau), \ldots, X_{n_x}(t-\tau)\} \) — множество родителей переменной \( Y_j(t) \) из \( \mathbf{X} \),
	- \( f_i \) и \( g_j \) — детерминированные функции, описывающие зависимость,
	- \( \varepsilon_{X_i}(t) \) и \( \varepsilon_{Y_j}(t) \) — шумовые компоненты.
	
	Оптимизационная задача:
	\[
	\min_{G_{XY}, G_{YX}} \mathcal{L}(\mathbf{X}, \mathbf{Y} \mid G_{XY}, G_{YX}) + \lambda_1 \mathcal{R}(G_{XY}, G_{YX}) + \lambda_2 \mathcal{T}(G_{XY}, G_{YX}),
	\]
	где:
	- \( G_{XY} \) — граф зависимостей \( X_i \to Y_j \),
	- \( G_{YX} \) — граф зависимостей \( Y_j \to X_i \),
	- \( \mathcal{L}(\mathbf{X}, \mathbf{Y} \mid G_{XY}, G_{YX}) \) — правдоподобие наблюдаемых данных с учетом графов \( G_{XY} \) и \( G_{YX} \),
	- \( \mathcal{R}(G_{XY}, G_{YX}) \) — регуляризатор, штрафующий за сложность графов,
	- \( \mathcal{T}(G_{XY}, G_{YX}) \) — штраф за избыточную изменчивость графов во времени.
	
	\section{Предлагаемый подход}
	\begin{algorithm}[h!]
		\caption{Алгоритм вероятностного выявления влияния $\mathbf{X}\to\mathbf{Y}$ на основе ICA и ядерной оценки плотностей}
		\label{alg:ICA_KDE_Causal}
		
		\textbf{Входные данные:}
		\begin{itemize}
			\item Набор исходных наблюдений $\{\mathbf{X}_{\mathrm{raw}}(t)\}_{t=1}^T \subset \mathbb{R}^{d_X}$ (ЭЭГ).
			\item Набор исходных наблюдений $\{\mathbf{Y}_{\mathrm{raw}}(t)\}_{t=1}^T \subset \mathbb{R}^{d_Y}$ (ИИМ).
			\item Параметры: число независимых компонент $r$ для ICA, размерность вложения $E$ и временной лаг $\tau$, ядровые ширины сглаживания $h_X,\,h_Y$.
		\end{itemize}
		
		\textbf{Выходные данные:}
		\begin{itemize}
			\item Временной ряд (или процесс) $\{\gamma(t)\}_{t=1}^T$, где
			\[
			\gamma(t) = MI(M_{X, t}, M_{Y, t}) 
			\;=\; 
			\mathbb{E}_{M_{X, t}} D_{\mathrm{KL}}\Bigl(
			\widehat{p}\bigl(M_{Y,t}\,\big\vert\,M_{X,t}\bigr)
			\;\big\|\;
			\widehat{p}_{Y}\bigl(M_{Y,t}\bigr)
			\Bigr),
			\]
			измеряющий силу влияния $\mathbf{X}\to\mathbf{Y}$ в момент $t$.
		\end{itemize}
		
		\begin{enumerate}
			\item \textbf{Независимый анализ компонент (ICA).} 
			
			\(\displaystyle
			\mathbf{X}_{\mathrm{raw}}(t) \;=\; A\,\mathbf{S}(t),
			\quad
			\mathbf{S}(t)\in \mathbb{R}^{r}, 
			\quad
			A\in\mathbb{R}^{d_X \times r}.
			\)
			
			Вычислить оценки независимых компонент:
			\[
			\widehat{\mathbf{S}}(t)
			\;=\;
			\widehat{A}^{-1}\,
			\mathbf{X}_{\mathrm{raw}}(t).
			\]
			
			\item \textbf{Построение временных эмбеддингов для \(\mathbf{X}\) и \(\mathbf{Y}\).}
			
			Для каждого $t$ сформировать:
			\[
			M_{X,t}
			\;=\;
			\bigl(\widehat{\mathbf{S}}(t),\,\widehat{\mathbf{S}}(t-\tau),\,\dots,\widehat{\mathbf{S}}(t-(E-1)\tau)\bigr)
			\;\;\in\;\mathbb{R}^{rE},
			\]
			\[
			M_{Y,t}
			\;=\;
			\mathbf{Y}_{\mathrm{raw}}(t)
			\;\;\in\;\mathbb{R}^{d_Y E}.
			\]
			
			\item \textbf{Ядерная оценка (KDE) для маргинальных плотностей на эмбеддингах.}
			
			Определить оценки плотностей:
			\[
			\widehat{p}_X(u)
			\;=\;
			\frac{1}{T\,h_X^{m_X}}
			\;\sum_{t=1}^T 
			K\!\Bigl(\tfrac{\|u - M_{X,t}\|}{h_X}\Bigr),
			\quad
			u \in \mathbb{R}^{m_X},
			\]
			\[
			\widehat{p}_Y(v)
			\;=\;
			\frac{1}{T\,h_Y^{m_Y}}
			\;\sum_{t=1}^T 
			K\!\Bigl(\tfrac{\|v - M_{Y,t}\|}{h_Y}\Bigr),
			\quad
			v \in \mathbb{R}^{m_Y},
			\]
			где $m_X = rE,\; m_Y = d_Y,$ а $K(\cdot)$ --- ядровая функция.
			
			\item \textbf{Агрегация результатов и формирование временного ряда.}
			
			Сформировать последовательность
			\[
			\gamma(t)
			\;=\;
			\mathbb{E}_{M_{X, t}} D_{\mathrm{KL}}\Bigl(
			\widehat{p}\bigl(M_{Y,t}\,\vert\,M_{X,t}\bigr)
			\;\big\|\;
			\widehat{p}_Y\bigl(M_{Y,t}\bigr)
			\Bigr),
			\quad t = 1,\dots,T.
			\]
		\end{enumerate}
		
	\end{algorithm}
	
	\bibliographystyle{unsrtnat}
	\bibliography{references.bib}
	
\end{document}